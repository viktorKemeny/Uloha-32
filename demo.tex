% !TeX spellcheck = en_GB
\documentclass{acm_proc_article-sp} % alebo sig-alternate
\usepackage[english]{babel}
\usepackage[utf8]{inputenc}
\usepackage{url}
\usepackage{booktabs}
\setlength{\heavyrulewidth}{1.2pt}
\setlength{\lightrulewidth}{0.7pt}

\begin{document}



\title{Detecting Search Sessions Using Document Metadata and
	Implicit Feedback}

\numberofauthors{2}

\author{
\alignauthor
Tomáš Kramár\\
       \affaddr{Institute of Informatics and Software Engineering
}\\
       \affaddr{Faculty of Informatics and Information Technologies
}\\
       \affaddr{Slovak University of Technology
}\\
       \affaddr{Bratislava, Slovakia
}\\
       \email{kramar@fiit.stuba.sk}
\alignauthor
Mária Bieliková\\
       \affaddr{Institute of Informatics and Software Engineering
}\\
       \affaddr{Faculty of Informatics and Information	Technologies
}\\
       \affaddr{Slovak University of Technology
}\\
       \affaddr{Bratislava, Slovakia
}\\
       \email{bielik@fiit.stuba.sk}
}

\maketitle

\begin{abstract}
It has been shown that search personalization can greatly
benefit from exploiting user’s short-term context – user’s
immediate need and intent. However, this requires that the
search engine must be able to divide user’s activity into segments,
where each segment captures user’s single goal and
focus. Several different approaches to search session segmentation
exist, each considering different features of the
queries, but it may be helpful to also consider user’s implicit
feedback on the search results clicked in response to
the query. We propose a method for segmenting queries into
search sessions which is based on document metadata and
incorporates implicit feedback. Our approach also considers
multitasking, where user shifts her current interest, but afterwards
proceeds with the original task. We evaluated our
approach on manually segmented query log and compared
the results of our approach with results from other methods
and showed that using implicit feedback can improve the
performance of the segmentation task.

\end{abstract}

\category{H.3.3}{Information Search and Retrieval}{Query formulation,
	Selection process, Search process
}

\terms{Information retrieval}

\keywords{personalization, search, short-term contexts, search sessions,
	search session segmentation}


\section{Introduction}

Web contains an ever-growing amount of documents. Accessing
these document poses great difficulties, especially
after the rise of Web 2.0 where users have been given the
ability to create the content, which allows for more socialbased
approaches to personalization [1], but also contributes
to information overload. According to Technorati1
, a service
which tracks user-generated media, the content on the Web
is growing with a pace at 2 blog posts per second, and this
number does not include the growth of other content.

Search engines play a crucial role in accessing this amount
of content. Users interact with search engines by entering
few keywords, which describe their intent and expect the
machine to provide a list of relevant documents. This model
has several known disadvantages:

\begin{itemize}  
	\item the number of keywords is usually low, typically 1-3
	keywords [17] and this often leads to ambiguity and
	unclear intent;
	\item many of the words are ambiguous; a word “jaguar”
	can refer to an animal, a car and even has less-known
	meanings such as a game console or German battle
	tank;
	\item the queries are almost never accurate [9], they are either
	too generic or too specific, but almost never exactly
	aligned with the specific intent the user has in
	mind.
\end{itemize}

The combined impact of these problems leads to a conclusion
that finding the relevant document when we do not have
enough information about it is indeed a difficult task, both
for the user and the search engine.

To mitigate this problem, several approaches to search personalization
have been researched, each with the ultimate
aim to help users find the relevant content, without trying
to change how humans think, or work.

There is relevance feedback, query expansion, search intent
detection, alternative ranking schemes and many others.
These techniques leverage and act upon some form of search
context. Generally, the term context refers to attributes of
the environment [7], such as location, time, or weather, but
in the domain of personalized search the term is commonly
used to describe user’s needs, goals and intents (e.g. [21, 27]).
Based on the time span that is used to build the search context,
the context may be long, or short-term.

Long-term search context is composed of the goals and intents
that can be recognized by observing the complete user
activity, beginning with the first known information about
the user and her activity.

Short-term search context is composed of the goals and intents
that the user has in the moment of search. These
represent the current focus and are obtained by observing
the user activity beginning in a recent point of time.

To be able to use short-term search context a personalization
system must know the exact moment the user changes
her intent, so that it can start and use a new context. The
task of detecting this change is referred to as search session
detection (segmentation). The term search session was
never formally defined in the literature and its meaning differs
in different works. In this work, we assume that search
session is a sequence of search related actions with the single
underlying informational intent, similarly to [22].

The goal of search session segmentation is to partition the
stream of user queries into segments of queries, where each
segment is the search session, i.e., holds the condition that
all queries that it contains are related to a single underlying
goal.

Several existing approaches to search session segmentation
exist, but they have various disadvantages. When a segmentation
approach acts solely on the features provided by the
query itself, the amount of understanding of the underlying
intent is quite limited. Therefore, many approaches also
consider the documents that were clicked in response to the
query, but these approaches do not evaluate user’s feedback
that is implicitly left in each document. Many existing approaches
also do not consider interruptions in Web browsing
and multitasking (i.e. having multiple intents).

In this work we aim to contribute to the area of search session
segmentation by matching the queries with the metadata
of the documents clicked from the search results to get
better insight into the purpose of the query by aggregating
more data than only the query itself provides. We also
evaluate the level of page usefulness for the particular query
by collecting and analyzing the implicit feedback indicators
that the user provides for each page view. Our approach
also considers user interruptions and is able to separate intermingled
sessions and reconnect interrupted sessions.

The paper is structured as follows. Section 2 describes the
related work done in the area of search session segmentation.
In Section 3 we describe our data collection methodology.
Section 4 describes the proposed method for search
sessions segmentation. Experimental results are described
in Section 6.
\bibliographystyle{abbrv}
\bibliography{1Technorati, http://technorati.com/}

\balancecolumns

\end{document}
