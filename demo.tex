% !TeX spellcheck = en_GB
\documentclass{acm_proc_article-sp} % alebo sig-alternate
\usepackage[english]{babel}
\usepackage[utf8]{inputenc}
\usepackage{url}
\usepackage{cite}
\usepackage{booktabs}
\usepackage{graphicx}
\setlength{\heavyrulewidth}{1.2pt}
\setlength{\lightrulewidth}{0.7pt}

\begin{document}



\title{Detecting Search Sessions Using Document Metadata and
	Implicit Feedback}

\numberofauthors{2}

\author{
\alignauthor
Tomáš Kramár\\
       \affaddr{Institute of Informatics and Software Engineering
}\\
       \affaddr{Faculty of Informatics and Information Technologies
}\\
       \affaddr{Slovak University of Technology
}\\
       \affaddr{Bratislava, Slovakia
}\\
       \email{kramar@fiit.stuba.sk}
\alignauthor
Mária Bieliková\\
       \affaddr{Institute of Informatics and Software Engineering
}\\
       \affaddr{Faculty of Informatics and Information	Technologies
}\\
       \affaddr{Slovak University of Technology
}\\
       \affaddr{Bratislava, Slovakia
}\\
       \email{bielik@fiit.stuba.sk}
}

\maketitle

\begin{abstract}
	
It has been shown that search personalization can greatly
benefit from exploiting user’s short-term context – user’s
immediate need and intent. However, this requires that the
search engine must be able to divide user’s activity into segments,
where each segment captures user’s single goal and
focus. Several different approaches to search session segmentation
exist, each considering different features of the
queries, but it may be helpful to also consider user’s implicit
feedback on the search results clicked in response to
the query. We propose a method for segmenting queries into
search sessions which is based on document metadata and
incorporates implicit feedback. Our approach also considers
multitasking, where user shifts her current interest, but afterwards
proceeds with the original task. We evaluated our
approach on manually segmented query log and compared
the results of our approach with results from other methods
and showed that using implicit feedback can improve the
performance of the segmentation task.

\end{abstract}

\category{H.3.3}{Information Search and Retrieval}{Query formulation,
	Selection process, Search process
}

\terms{Information retrieval}

\keywords{personalization, search, short-term contexts, search sessions,
	search session segmentation}


\section{Introduction}

Web contains an ever-growing amount of documents. Accessing
these document poses great difficulties, especially
after the rise of Web 2.0 where users have been given the
ability to create the content, which allows for more socialbased
approaches to personalization [1], but also contributes
to information overload. According to Technorati1
, a service
which tracks user-generated media, the content on the Web
is growing with a pace at 2 blog posts per second, and this
number does not include the growth of other content.

Search engines play a crucial role in accessing this amount
of content. Users interact with search engines by entering
few keywords, which describe their intent and expect the
machine to provide a list of relevant documents. This model
has several known disadvantages:

\begin{itemize}  
	\item the number of keywords is usually low, typically 1-3
	keywords [17] and this often leads to ambiguity and
	unclear intent;
	\item many of the words are ambiguous; a word “jaguar”
	can refer to an animal, a car and even has less-known
	meanings such as a game console or German battle
	tank;
	\item the queries are almost never accurate [9], they are either
	too generic or too specific, but almost never exactly
	aligned with the specific intent the user has in
	mind.
\end{itemize}

The combined impact of these problems leads to a conclusion
that finding the relevant document when we do not have
enough information about it is indeed a difficult task, both
for the user and the search engine.

To mitigate this problem, several approaches to search personalization
have been researched, each with the ultimate
aim to help users find the relevant content, without trying
to change how humans think, or work.

There is relevance feedback, query expansion, search intent
detection, alternative ranking schemes and many others.
These techniques leverage and act upon some form of search
context. Generally, the term context refers to attributes of
the environment [7], such as location, time, or weather, but
in the domain of personalized search the term is commonly
used to describe user’s needs, goals and intents (e.g. [21, 27]).
Based on the time span that is used to build the search context,
the context may be long, or short-term.

Long-term search context is composed of the goals and intents
that can be recognized by observing the complete user
activity, beginning with the first known information about
the user and her activity.

Short-term search context is composed of the goals and intents
that the user has in the moment of search. These
represent the current focus and are obtained by observing
the user activity beginning in a recent point of time.

To be able to use short-term search context a personalization
system must know the exact moment the user changes
her intent, so that it can start and use a new context. The
task of detecting this change is referred to as search session
detection (segmentation). The term search session was
never formally defined in the literature and its meaning differs
in different works. In this work, we assume that search
session is a sequence of search related actions with the single
underlying informational intent, similarly to [22].

The goal of search session segmentation is to partition the
stream of user queries into segments of queries, where each
segment is the search session, i.e., holds the condition that
all queries that it contains are related to a single underlying
goal.

Several existing approaches to search session segmentation
exist, but they have various disadvantages. When a segmentation
approach acts solely on the features provided by the
query itself, the amount of understanding of the underlying
intent is quite limited. Therefore, many approaches also
consider the documents that were clicked in response to the
query, but these approaches do not evaluate user’s feedback
that is implicitly left in each document. Many existing approaches
also do not consider interruptions in Web browsing
and multitasking (i.e. having multiple intents).

In this work we aim to contribute to the area of search session
segmentation by matching the queries with the metadata
of the documents clicked from the search results to get
better insight into the purpose of the query by aggregating
more data than only the query itself provides. We also
evaluate the level of page usefulness for the particular query
by collecting and analyzing the implicit feedback indicators
that the user provides for each page view. Our approach
also considers user interruptions and is able to separate intermingled
sessions and reconnect interrupted sessions.

The paper is structured as follows. Section 2 describes the
related work done in the area of search session segmentation.
In Section 3 we describe our data collection methodology.
Section 4 describes the proposed method for search
sessions segmentation. Experimental results are described
in Section 6.

\section{RELATED WORK}

The most widely used approach to search session segmentation
is to compare temporal distance of the queries. If two
queries were issued with a time difference larger than a predefined
threshold, it is assumed that a new session started,
and the existing session is split at that place. This technique
was first described in [4], establishing the threshold
of 25.5 minutes. They measured an average time between
page requests (9.5 minutes) and added a 1.5 standard deviation.
This approach only makes sense for normal distribution,
while surfing on the Web shows properties of a long-tail
distribution. The established 25.5 minutes are part of the
long-tail distribution and it wouldn’t make much difference
if 20, or 40 minutes were used instead [14]. He et al. [13] experimented
with various cutoff settings and established that
the optimal cutoff time is between 10 and 15 minutes. Due
to its simplicity both in concept and implementation, this
technique is widely used and there are modification ranging
from use of 5 minutes cutoff [8], through 30 minutes cutoff
[24] to a per-user cutoff [23].

\begin{itemize}  
	\item it is unable to detect sessions which are split within
	a short time – if the search intent changes quickly,
	without sufficient time passing between consecutive
	queries, the queries are falsely considered as part of
	a single session, yet they might bear a different underlying
	intent;
	\item the long time between searches may not mean that the
	user’s intent changed – this is a well known downside
	of any time-window based method. The longer pause
	between queries may be caused by several factors, such
	as reading long article, breaks, or any other interruptions
	that are commonly encountered nowadays [5].
\end{itemize}

Another approach uses lexical distance of the queries. This
idea compares content of two queries to detect if the intent
has changed. Example of this approach is stated in [16].
The main drawback of this method is that it leads to a high
amount of false positives. There are many instances where
two queries are completely dissimilar (they share no common
words), yet their underlying intent is the same. Consider
a user searching for “IR” and “information retrieval”
afterwards. Using the lexical distance approach would incorrectly
yield two separate sessions.

Method described in [11] combines both temporal and lexical
distances. They use vector-space representation, where each
pair of following queries is placed in the space. If the query
pair fits into the space bounded by the subplane delimited
by two edge cases

\begin{itemize} 
	\item two parallel, but dissimilar queries,
	\item the same queries, but executed long time apart
\end{itemize}

it is considered an extension of the current session. Although
this combination can achieve better results than each of the
methods alone, it is still prone to the aforementioned problems.

As stated, related queries do not provide sufficient data to
match similar intents, but there are approaches that use
signals from the retrieved documents. In [25], authors used
vectors of titles and snippets of 50 top-ranked documents
for the query. The session is split as determined by comparing
cosine similarities of two following queries. Another
approach uses document keywords. In [6], authors extract
document’s keywords using the TF.IDF scheme and map
them to ODP categories. The session is then split when the
ODP category changes. This approach however fails to account
for user’s real interests. When the user enters a query
and clicks and views some documents only to realize that the
results are wrong and the query needs to be reformulated,
this approach has already used these faulty results to decide
upon session segmentation.

Jones and Klinker [18] trained a binary classifier to detect
whether two queries are part of the same search task or not,
using a set of syntactic, temporal, query log and web search
features. They have obtained best results by incorporating
a vector derived from the first 50 results for the given query,
but this approach again suffers from the query ambiguity.

Currently, the arguably best approach is grouping related
queries into sessions by using a clustering methods. The
key feature of the approach is wikification of the queries,
i.e., building a relevance vector in a high dimensional concept
space of Wikipedia articles [22]. The relevance vector
describes relevance of the query terms for each Wikipedia
article. Although this approach gives good result on the
manually annotated set of circa 1300 queries, it is strongly
dependent on the Wikipedia articles. Given a query which
does not contain a term present on Wikipedia, this approach
does not yield any benefit over other methods. Our method
is not dependent on other external and limited knowledge
base, instead, it depends only on clicked search results which
follow naturally after every search.

Most of the existing approaches also do not deal with multitasking.
Multitasking might be present in several forms,
mainly:

\begin{itemize} 
	\item parallel browsing, when a user is having multiple goals
	at the same time and working on them at the same
	time by means of having multiple browsers, or multiple
	browser tabs, or
	\item browsing with interruptions, when a user is having a
	single goal but is interrupted, switches the goal for a
	while and returns back to the original goal afterwards.
	The interruption may not be forced, it may be the
	user’s conscious decision to abandon the current task
	and concentrate on different goal.
\end{itemize}

The practical effect of multitasking on search session segmentation
is that it is causing the sessions to be disconnected.
Single search session is interrupted by queries that
are part of another session and then the session continues,
as the user returns to the original goal.

There are mixed reports on the presence of multitasking on
the Web. According to [15] only 25% of users multitask.
When searching, the multitasking is practically nonexistent
for navigational and transactional queries [28], it is only
present in 6-15% of ambiguous queries. Older study [3] reports
that only 1% of all sessions are diconnected. Lucchese
et al. analyzed AOL query log and report that multitasking
in search exists [22].

In [3] a method to detect disconnected sessions is proposed.
Their approach is executed in two steps: in the first phase,
sessions are segmented using a cutoff time, in the second
step, all queries in each session are compared to build a similarity
graph where the transitive closure is computed. All
queries connected by the closure are considered part of the
same session. This approach again combines the disadvantages
of the approaches it merges – temporal and lexical
distance. In [22] the queries are connected using clustering
methods, so the multitasked queries are connected naturally.

Our work extends and differs from existing research in several
important ways


\begin{itemize} 
	\item First, we only consider the documents that the user
	deemed useful. To measure the usefulness, we partially
	rely only on the clickthrough data and only consider
	the search results that the user actually viewed, but
	we also use implicit feedback indicators to weight the
	contribution of each of the documents.
	\item The search sessions are segmented by comparing their
	metadata (both machine and human extracted), and
	to increase the chance of a successful match, we extend
	the metadata with ConceptNet relations.
	\item Our approach also considers multitasking and user interruptions
	– we maintain a stack of short contexts
	(search sessions) and when the query fits within a previously
	abandoned session, this session can be popped
	from the stack and continued.
\end{itemize}

\section{DATA COLLECTION}

Our work is based on the idea of considering only the documents
that the user found useful, therefore we need a way
to collect user’s actions on search engine and implicit feedback
on viewed documents [10]. In order to gain detailed
insight into users actions and to collect precise and rich implicit
feedback data, we leverage a personalized proxy server,
called PeWe Proxy. The collection process is described in
more detail in Fig. 1.

The personalized proxy server2
acts as a regular proxy, sitting
between the user and the Web [20]. All user’s requests
are handled by the proxy first and then forwarded to the
target server. Similarly, the response from Web server has
to pass through the proxy. This means that the proxy has
access to complete message processing between client and
the server and can even modify the actual content of the
request/response.

We are aware, that using proxy server in a production search
engine is not a realistic option. We are using proxy server to
gain detailed insight into implicit feedback on every accessed
page. Some level of implicit feedback can be inferred by
analyzing timestamps and patterns in query logs [26].

\includegraphics{img-1.jpg}

\subsection{Implicit feedback collection}

We use the proxy server to inject a tracking JavaScript into
each page, which collects the following implicit feedback signals:

\begin{itemize} 
	\item \textit{real time spent on page} – the actual time spent on page
	is measured in series of short time windows. If there
	is an observed activity within the page (i.e., mouse
	movement, scrolling, clicking, writing) during the span
	of the window, the length of the window is added to the
	total time on page. In this case, we used the window
	of 4 seconds;
	\item number of clicks, scrolls and copying into clipboard;
\end{itemize}

\subsection{Metadata collection}

Besides the implicit feedback collection, we use PeWeProxy
to create logs of user activity, i.e., we track the documents
that the user visited and to associate each log with its corresponding
implicit feedback signals.

Each document is processed and following types of metadata
are extracted:

\begin{itemize} 
	\item \textit{keywords} – we extract keywords automatically by using
	the tagthe.net and Alchemy Orchestr8 Web services
	and JATR library;
	\item \textit{tags} – we use human created directory of tags, the Web
		service delicious.com;
	\item \textit{named entities} extracted by using \textit{OpenCalais} Web
		service;
	\item \textit{categories} from the human-maintained project ODP
\end{itemize}

All metadata is extracted by using a wrapper Web service
Metall3
, which strips the document from auxiliary content
(ads, menus, header, footer, etc.), leaving only the main text
of the document, which is translated into English and the
described services are used to extract metadata. We translate
the documents into English, as the available extraction
services work best with English text. The automatic translation
is far from perfect, but the basic informational structure
of the text is retained and as our previous experiments show,
the quality of metadata extracted from translated texts is
satisfactory [2].






\begin{itemize} 
	\item
	\item
\end{itemize}


\bibliographystyle{abbrv}
\bibliography{}

\balancecolumns

\end{document}
